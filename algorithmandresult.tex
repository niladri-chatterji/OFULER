\subsection{Algorithm} 

\begin{algorithm}
\label{algorithm:OFULER}
\caption{\algname}
\For{$t = 1, \ldots, n$} {
    \textit{UCB: } $i_t \in \argmax_{\iinrange{1}{K}} \left[\max_{\mu \in \mathcal{C}_t^{s}(i)} \mu\right]$. \\
    
    \textit{OFUL: }$j_t,\tilde{\Omega}_t \in \argmax_{\iinrange{1}{K}, \Omega \in \mathcal{C}_t^c} \langle X_{i,t},\Omega \rangle$. \\

    \If{$t > \tstar$ and 
        \begin{align} \label{test:complextest} \sum_{s=1}^t \langle X_{j_s,s}, \tilde{\Omega}_s \rangle - \sum_{s=1}^t g_{i_s,s} \ge 100 \log(n) d\sqrt{t}.\end{align}}{
        
        }
        
}

\end{algorithm}


\note{clean algorithm up. Add intuition behind the pieces. Inspired by combining SAO with OFUL and UCB. Why does the test make sense.}

\begin{theorem}\label{thm:mainregretbound}
If we run \algname then we have the guarantee with high probability that,
\begin{itemize}
    \item Under the \emph{Simple Model} we have $\psued{n} \le \tilde{\mathcal{O}}\left(\sqrt{K n}\right)$.
    \item Under the \emph{Complex Model} we have $\regret{n} \le \tilde{\mathcal{O}}\left(d\sqrt{K n}\right).$
\end{itemize}
\end{theorem}

